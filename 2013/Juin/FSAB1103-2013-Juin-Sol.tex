\documentclass[11pt,a4paper]{article}
\usepackage[utf8]{inputenc}
\usepackage[T1]{fontenc}
\usepackage{lmodern}
\usepackage[french]{babel}
\usepackage[hidelinks]{hyperref}
%\usepackage{hyperref}
\usepackage{url}
\usepackage{cite}
\usepackage[babel=true]{csquotes} 
\usepackage{amsmath}
\DeclareMathOperator{\arcsinh}{arcsinh}
\usepackage{amsthm}
\usepackage{amssymb}
\usepackage[version=3]{mhchem}
\usepackage{vmargin}
\usepackage{pgfplots}
\usepackage{enumerate}

%\usepackage{subfig}
% subfig is deprecated see
% http://en.wikibooks.org/wiki/LaTeX/Floats,_Figures_and_Captions#Subfloats
\usepackage{caption}
\usepackage{subcaption}

% Numbers and units
\usepackage[squaren, Gray]{SIunits}
\usepackage{sistyle}
\usepackage[autolanguage]{numprint}
%\usepackage{numprint}
\newcommand\si[2]{\numprint[#2]{#1}}
\newcommand\np[1]{\numprint{#1}}

\usepackage{ifthen}
%\input{config.tex}

\ifthenelse{\isundefined{\Sol}}{\def\Sol{true}}{}

\newcommand{\solution}[1]
{\ifthenelse{\equal{\Sol}{true}}{\subsection*{Solution}#1}{}}

\DeclareMathOperator{\newdiff}{d} % use \dif instead
\newcommand{\dif}{\newdiff\!}
\newcommand{\fpart}[2]{\frac{\partial #1}{\partial #2}}
\DeclareMathOperator{\res}{Res}
\DeclareMathOperator{\arctanh}{arctanh}
\newcommand{\ffpart}[2]{\frac{\partial^2 #1}{\partial #2^2}}
\newcommand{\fdpart}[3]{\frac{\partial^2 #1}{\partial #2\partial #3}}
\newcommand{\fdif}[2]{\frac{\dif #1}{\dif #2}}
\newcommand{\ffdif}[2]{\frac{\dif^2 #1}{\dif #2^2}}
\newcommand{\constant}{\ensuremath{\mathrm{cst}}}

\newcommand{\rappelscomplexesbase}{
  \textbf{Rappels:} $\sin(w) = \frac{1}{2i}\left(e^{iw}-e^{-iw}\right)
  = \frac{1}{i}\sinh(iw)$ et
  $\cos(w) = \frac{1}{2}\left(e^{iw}+e^{-iw}\right) = \cosh(iw)$.
}
\newcommand{\rappelscomplexes}{
  \rappelscomplexesbase
  On a aussi, pour $|Z| < 1$, que
  $\frac{1}{1+Z} = 1 - Z + Z^2 - Z^3 + \ldots$
}

\newcommand{\HRule}{\rule{\linewidth}{0.5mm}}

\newcommand{\hypertitle}[3]{
\usepackage{hyperref}
{\renewcommand{\and}{\unskip, }
\hypersetup{pdfauthor={#3},
            pdftitle={FSAB1103 : Correction de l'examen de #2 #1},
            pdfsubject={Math\'ematique}}
}
\ifthenelse{\equal{\Sol}{true}}%
{\title{FSAB 1103 : Correction de l'examen de #2 #1}}%
{\title{FSAB 1103 : Énoncé de l'examen de #2 #1}}
\author{#3}

\begin{document}

\maketitle

\paragraph{Informations importantes}
Ce document a été réalisé par les étudiants susnommés et
est donc à prendre avec des ``pincettes''.
Si vous observez des fautes ou avez des suggestions à faire,
n'hésitez pas à nous contacter.
On peut retrouver le code source à l'adresse suivante
\begin{center}
  \url{https://github.com/blegat/math3-examens}.
\end{center}
On y trouve aussi le contenu du \texttt{README} qui contient de plus
amples informations, vous êtes invité à le lire.
}


\hypertitle{2013}{Juin}{De Wolf Christophe\and Legat Beno\^it}

\paragraph{Resource utile}
\url{http://www.forum-epl.be/viewtopic.php?t=10154}

\section*{Question 1}
On considère le modèle de trafic routier LWR
avec la condition initiale suivante.
Soit un feu rouge situé en $x = 0$.
Pour les $-d \leq x < 0$,
la densité de trafic est maximale et les voitures sont,
de ce fait, à l'arrêt.
Pour les $x > 0$ et $x < -d$,
il n'y a pas de voitures.
En $t = 0$, le feu passe au vert.
\begin{enumerate}
  \item Transformer en une équation du transport de $\rho(x,t)$.
  \item Dessiner les caractéristiques de $\rho(x,t)$.
  \item Donner $x(t)$ la trajectoire d'une voiture située en $-d$
    (la fin de la file quand le feu passe au vert).
  \item Au bout de combien de temps la dernière voiture passe le feu ?
\end{enumerate}

\solution{
}

\section*{Question 2}
On considère l'EDP de Laplace pour $u(r,\theta)$ :
$$\nabla ^2u =
\ffpart{u}{r} +\frac{1}{r} \fpart{u}{r} + \frac{1}{r^2}\ffpart{u}{\theta} = 0$$
dans un cercle de rayon $a$ avec une coupure en $\theta = 0$.
% FIXME est-ce que ça a du sens de mettre coupure ici ?
Sur le pourtour du cercle,
on impose que $u(a, \theta) = g(\theta)$.
Sur la coupure en $\theta = 0$,
on impose que $u(r, 0) = 0$ d'un côté et
$\fpart{u}{\theta}(r, 2\pi) = \frac{u_0}{2\pi}$ de l'autre.
\begin{enumerate}
  \item Faites un dessin du domaine dans le plan physique $(x,y)$ et dans le
    système $(r,\theta)$ et
    indiquez les conditions aux limites sur chaque bord.
  \item Obtenez par la méthode de séparation des variables,
    la solution sous forme d'un développement en série.
    Écrivez clairement les intégrales à effectuer
    pour obtenir les coefficients du développement.
\end{enumerate}

\solution{
}

\section*{Question 3}
Idem que la question 3 de Janvier 2012.

\section*{Question 4}
Calculez l'intégrale suivante
$$\int_0^\infty\frac{\cos(x)}{x^2+a^2}\dif x$$
où $a$ est un réel strictement positif par la méthode des résidus.

\solution{
}

\end{document}
